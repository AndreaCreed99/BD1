\documentclass[12pt,a4paper]{book}
\usepackage[utf8]{inputenc}
\usepackage{geometry}
\usepackage{fancyhdr}
\usepackage{graphicx}

\title{Esercitazioni SQL}
\author{Andrea Giacalone}
\date{A.A 2022/23}



\begin{document}
\maketitle

\pagestyle{fancy}
\fancyhf{}
\fancyhf[EHC]{Esercitazioni SQL}
\fancyhf[EFC]{\thepage}
\fancyhf[OFC]{\thepage}
\fancyhf[OHC]{Esercitazioni SQL}


\chapter{Esercitazione 2}

14) Trovare le informazioni sugli studenti e sugli esami che hanno sostenuto. Devono essere inclusi nel risultato anche gli studenti che non hanno sostenuto esami.


Che tabelle mi servono? STUDENTE ed ESAME. 

NB: Quando ci viene richiesto di trovare studenti, esami, professori, ... senza alcuna specifica ulteriore abbiamo due possibilità:
    SELECT * FROM
    estrarre la chiave primaria di quella tabella.

Che tipo di JOIN devo fare?

non è possbile join e basta perchè farebbe solo intersezione fra tabelle e non includerebbe gli studenti che non hanno sostenuto esami.

Dunque dovrò fare LEFT JOIN su \emph{STUDENTE}.

NB: a seconda dell'ordine che mettiamo nella clausola FROM, dovremo fare Left o Right Join.

    Soluzione:
        SELECT
        FROM Studente LEFT JOIN Esame ON
            Matricola = MatricolaStudente\\
        NB: soluzione potrebbe essere ottimizzata perchè in questo modo ottengo
            due campi Matricola

        WHERE: non ho bisogno di imporre ulteriori condizioni

\section{Il valore NULL}
I valori nulli possono essere usati in 3 diverse situazioni:
    1) il valore è applicabile ma sconosciuto -- valore ignoto al progettista
    2) non posso applicare un valore -- es. data del parto effettuato da un maschio
    3) non so se il valore è applicabile  -- es. valutazione di una serie di cui non si sa ancora eventuale data di uscita.

Come trattiamo questo valore NULL?\\
Lo utilizziamo quando vogliamo imporre una condizione di JOIN quindi se vogliamo estrarrre tutte le tuple di cui non sappiamo una valutazione:

    attributo IS NULL errore: attributo = NULL

NB: SQL ragiona con una logica a tre valori: TRUE (T), FALSE(F), UNKNOWN (U).
copiare tabella riassuntiva da slides soluzioni...

\subsubsection{Esercizio 15}: Calcolare il voto medio degli esami dei corsi da 5 crediti della facoltà 'ngegneria' tenuti da professori di nome 'Piero' la cui città di nome 'Piero' la cui ciità di residenza non è nota.

Che tabelle mi servono? 
ESAME, INSEGNANTE e CORSO

SELECT AVG(Voto)
FROM (Esame E JOIN Corso C on CodiceCorso = Codice) JOIN Insegnante I ON MatricolaProf = Matricola
WHERE NumeroCrediti = 5 AND Facoltà = 'Ingegneria' AND Nome = 'Piero' AND Città IS NULL

ATTENZIONE: C'è un errore in questa query: bisogna specificare la tabella da dove estraiamo l'attributo Nome poicheè lo troviamo sia in Insegnante che in Corso.

SELECT AVG(Voto)
FROM (Esame E JOIN Corso C on CodiceCorso = Codice) JOIN Insegnante I ON MatricolaProf = Matricola
WHERE NumeroCrediti = 5 AND Facoltà = 'Ingegneria' AND I.Nome = 'Piero' AND Città IS NULL



\section{Raggruppameto e Target List}
Questa query presenta un errore:
nella SELECT ha C.Città
nella GROUP BY ha O.CodCli
Abbiamo detto che se ho un \emph{raggruppamento}, nella SELECT possiamo solo mettere solo o aggregati o gli attributi che figurano nel raggruppamento; tuttavia a prima vista sembrerebe non esserci poichè per ogni cliente ho una sola città, ma SQL in ogni caso non può farlo.

Come si risolve questo errore se io voglio mantenere la Città nella SELECT?
Aggiungo nel GROUP BY anche C.Città: cambia qualcosa? In questo caso no, poichè vale il discorso fatto dell'unicità della città per ogni cliente.

\subsection{Esercizio 16}: trovare matricola, cognome, nome degli studenti di Milano che hanno superato esami per almeno un totale di almeno 20 crediti.

SOLUZIONE 1:
SELECT 
FROM Student S, Corso C, Esame E
WHERE Matricola = MatricolaStudente AND CodiceCorso = Codice AND SUM(NumeroCrediti) >= 20 
GROUP BY Matricola, Cognome, S.Nome

Non va bene: faccio il Join fra Studente, Esame e Corso. Poi applico la WHERE, ma la WHERE lavora riga per riga (dunque SUM(NumeroCrediti = 12); non potrò mai avere nella WHERE aggregati.
Gli aggregati nella SELECT o nella HAVING. Perchè una volta fatto il gruppo posso applicare la condizione.

Dove metto Città = Milano? Ho due possibilità WHERE o HAVING.
Sbagliato nella HAVING perchè nella HAVING non ho più singole righe ma gruppi, quindi la città di un gruppo per SQL è una cosa che non esiste. Quando io ho gruppi posso fitrare solo per aggregati.

SOLUZIONE 2:
SELECT Matricola, Cognome, S.Nome
FROM Studente S, Corso C, Esame E
WHERE Matricola = MatricolaStudente AND CodiceCrso = Codice AND Città = 'Milano'
GROUP BY Matricola, Cognome, S.Nome
HAVING SUM(NumeroCrediti) >= 20

\section{Set Queries}
Sono tre e sono: union, intersect,except.
Si usano quando voglio aggregare i risultati di più query. Essendo operatori insiemistici, i risultati vengono eliminati, a meno che non venga specificata l'opzione \emph{all}. 
NB: i risultati delle query da concatenare devono essere compatibili ossia:
    stesso numero di colonne
    a parità posizionale delle colonne, stesso dominio

\subsubsection{Esercizio 17}: Trovare la matricola degli studenti che hanno preso almeno un 30 e almeno un 30 e lode, supponendo che il voto 30 e lode sia memorizzato come 33-


SOLUZIONE 1: 
SELECT *
FROM Esame
WHERE Voto = 30 AND Voto = 33

E' sbagliata perchè vi è una contraddizione: no contemporanemente 30 e 33

SOLUZIONE 2:
SELECT MatricolaStudente
FROM Esame
WHERE Voto = 30 OR Voto = 33

E' sbagliata perchè verrebbe incluso anche chi ha preso un 30 e poi resto tutti 18.

Posso utilizzare allora gli operatori insiemistici.

SELECT MatricolaStudenti
FROM Esame
WHERE Voto = 30
INTERSECT
SELECT MatricolaStudenti
FROM Esame
WHERE Voto = 33

NB: se uno studente avesse preso più volte 30 e un 30L il risultato della mia query sarà lo studente ripetuto una sola volta perchè non tengo i duplicati.

NB: i risultati delle query sono compatibili poichè sto confrontando numero uguali di colonne e con attributi che corrispondentemenre alla posizione hanno stesso dominio (tipo).

\subsubsection{Esercizio 18}
Trovare i cognomi dei professori che non sono anche cognomi di studenti di Milano.\\
\\
SOLUZIONE:\\
\\
SELECT Cognome\\
FROM Insegnante\\
EXCEPT\\
SELECT Cognome\\
FROM Studenti\\
WHERE Città = 'Milano'\\

\section{Sub-Query}
Nella clausole WHERE e HAVING possono comparire predicati che confrontano un attributo (o un'espressione di attributi) con il risultato di una intera query SQL: dunque non più confronto con singola stringa, ..., ma con intero risultato.\\

Operator: uno dei vari operatori (NB: il simbolo di disuguaglianza è <>)
ANY: si mette quando il predicato è vero se almeno una riga restituita dalla query QuerySQL soddisfa il confronto 
ALL: il predicato è vero se tutte le righe resistuite dalla query QuerySQL soddisfano il confronto.

\subsection{Compatibilità delle query}
L'attributo della query più esterna deve essere compatibile con l'attributo della query più interna (numero uguale e dominio uguale).

\subsection{Visibilità degli alias}
NB: Gli alias sono sia quelli che si creano con AS e anche la visibilità dei campi: i campi delle query esterne li posso utilizzare anche nelle query interne: dunque posso fae sia binding fra query esterna e interna ma posso anche introdurre ambiguità, perchè non so poi a quale tabella appartengono.\\
Si risolve creando alias.

\subsubsection{Esercizio 19}
Trovare i codici e i nomi dei corsi con il massimo numero di crediti.\\
\\
SOLUZIONE 1:\\
\\
SELECT Codice,Nome\\
FROM Corso\\
WHERE MAX(NumeroCrediti)\\
NB:sbagliato così come mettere WHERE NumeroCrediti = MAX(NumeroCrediti) sempre perchè la WHERE viene valutata riga per riga.\\
\\
SOLUZIONE 2:\\
\\
SELECT Codice, Nome\\
FROM Corso\\
WHERE NumeroCrediti = (SELECT MAX(C.NumeroCrediti FROM Corso C))

NB: poichè la WHERE viene valutata riga per riga, la subquery verrà eseguita più volte, dunque poco performante
NB: se ho più corsi con il max numero di crediti, funziona perchè li terrò tutti che soddisfano.

SOLUZIONE 3:\\
SELECT Codice, Nome\\
FROM Corso\\
WHERE NumeroCrediti >= ALL (SELECT C.Numero Crediti FROM Corso C)
NB: questa volta sto estraendo tutti i numero crediti dei corsi e chiedo che sia maggiore uguale(attenzione a porre l'uguale perchè il max è compreso in questo numero altrimenti restituirebbe l'insieme vuoto)

\subsubsection{Esercizio 20}
Troavre i codici e i nomi di tutti i corsi ad eccezione di quelli con il minor numero di crediti.\\
\\
\textbf{SOLUZIONE 1}:\\
\\
SELECT Codice, Nome\\
FROM Corso\\
WHERE NumeroCrediti <> (SELECT MIN(C.NumeroCrediti) FROM Corso C)\\
\\
\textbf{SOLUZIONE 2}:\\
\\
SELECT Codice,Nome\\
FROM Corso\\
WHERE NumeroCrediti > ANY(SELECT C.NumeroCrediti FROM Corso C)\\
\\
\textbf{SOLUZIONE 3}:\\
\\
SELECT Codice, Nome\\
FROM Corso\\ 
EXCEPT\\
SELECT Codice, Nome\\
FROM Corso\\
WHERE NumeroCrediti = (SELECT MIN(C.NumeroCrediti) FROM Corso C)\\

\section{Operatori IN e NOT IN}
L'operatore IN è equivalente a = ANY.
L'operatore NOT IN è equivalente a <> ALL.

\subsubsection{Esercizio 21}
Trovare matricola nome  ecognome degli studeneti che hanno preso almeno un voto maggiore di 20 in esami con nome del corso che inizia con 'AN' e la cui quarta lettera è 'L'.\\
\\
SOLUZIONE 1:\\
\\
SELECT Matricola, Nome, Cognome\\
FROM Studente\\
WHERE Matricola IN (SELECT MatricolaStudente FROM Esame, Corso C WHERE CodiceCorso = Corso AND Voto> 20 AND C.Nome LIKE "AN_L%")

NB: la cui matricola è tra... -- WHERE Matricola IN\\
\\
SOLUZIONE 2:\\
\\
SELECT Matricola, Nome, Cognome, S.Nome
FROM Studente S, Esame, Corso C
WHERE CodiceCorso = Codidce AND Matricola = MatricolaStudente AND Voto > 20 AND C.Nome LIKE "AN_L%")

NB: in questa soluzione però otteniamo duplicati; per farla in maniera identica alla prima senza però utilizzare subquery dobbiamo:

SOLUZIONE 3:\\
\\
SELECT DISTINCT Matricola, Nome, Cognome, S.Nome
FROM Studente S, Esame, Corso C
WHERE CodiceCorso = Codidce AND Matricola = MatricolaStudente AND Voto > 20 AND C.Nome LIKE "AN_L%")

Con parola chiave DISTINCT eliminiamo i duplicati 

\section{Operatore EXISTS}
Questo operatore, che di fatto mi permette di andare a predicare e per questo andrà nella clausola WHERE o HAVING, mi restituirà vero se QuerySQL retsituisce almeno una riga non vuota.

NB: è possibile usare anche la sintassi NOT EXISTS

\subsubsection{Esercizio 22}
Trovare le informazioni di tutti i professori che hanno degli omonimi

SOLUZIONE 1:\\
\\
SELECT *\\
FROM Insegnante I1\\
WHERE EXISTS(SELECT * FROM Insegnate I2 HWERE I1.Nome = I2.Nome AND I1.Cognome = I2.Cognome AND I1.Matricola <> I2.Matricola)























\end{document}


